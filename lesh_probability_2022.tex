% arara: xelatex: {shell: yes}
% %arara: biber
% %arara: xelatex: {shell: yes}
% %arara: xelatex: {shell: yes}

\documentclass[12pt]{article}

\usepackage{hyperref} % гиперссылки

\usepackage{tikz} % картинки в tikz
\usetikzlibrary{arrows.meta} % tikz-прибамбас для рисовки стрелочек подлиннее

\usepackage{microtype} % свешивание пунктуации

\usepackage{array} % для столбцов фиксированной ширины

\usepackage{indentfirst} % отступ в первом параграфе

\usepackage{sectsty} % для центрирования названий частей
\allsectionsfont{\centering}

\usepackage{amsmath} % куча стандартных математических плюшек
\usepackage{amssymb} % символы
\usepackage{amsthm} % теоремки

\usepackage{comment} % добавление длинных комментариев

\usepackage[top=1cm, left=1.2cm, right=1.2cm, bottom=1cm]{geometry} % размер текста на странице

\usepackage{lastpage} % чтобы узнать номер последней страницы

\usepackage{enumitem} % дополнительные плюшки для списков
%  например \begin{enumerate}[resume] позволяет продолжить нумерацию в новом списке

\usepackage{caption} % что-то делает с подписями рисунков :)

\usepackage{qcircuit} % для рисовки квантовых диаграмм
\usepackage{physics} % бракеты

\usepackage{answers} % разделение условий и ответов в упражнениях


\usepackage{fancyhdr} % весёлые колонтитулы
\pagestyle{empty}



\usepackage{todonotes} % для вставки в документ заметок о том, что осталось сделать
% \todo{Здесь надо коэффициенты исправить}
% \missingfigure{Здесь будет Последний день Помпеи}
% \listoftodos — печатает все поставленные \todo'шки



\usepackage{booktabs} % красивые таблицы
% заповеди из докупентации:
% 1. Не используйте вертикальные линни
% 2. Не используйте двойные линии
% 3. Единицы измерения - в шапку таблицы
% 4. Не сокращайте .1 вместо 0.1
% 5. Повторяющееся значение повторяйте, а не говорите "то же"



\usepackage{fontspec} % что-то про шрифты?
\usepackage{polyglossia} % русификация xelatex

\setmainlanguage{russian}
\setotherlanguages{english}

% download "Linux Libertine" fonts:
% http://www.linuxlibertine.org/index.php?id=91&L=1
\setmainfont{Linux Libertine O} % or Helvetica, Arial, Cambria
% why do we need \newfontfamily:
% http://tex.stackexchange.com/questions/91507/
\newfontfamily{\cyrillicfonttt}{Linux Libertine O}

\AddEnumerateCounter{\asbuk}{\russian@alph}{щ} % для списков с русскими буквами
\setlist[enumerate, 2]{label=\asbuk*),ref=\asbuk*}

%% эконометрические сокращения
\DeclareMathOperator{\Cov}{Cov}
\DeclareMathOperator{\Arg}{Arg}
\DeclareMathOperator{\Corr}{Corr}
\DeclareMathOperator{\Var}{Var}
\DeclareMathOperator{\E}{\mathbb{E}}
\newcommand \hVar{\widehat{\Var}}
\newcommand \hCorr{\widehat{\Corr}}
\newcommand \hCov{\widehat{\Cov}}
\newcommand \cN{\mathcal{N}}
\let\P\relax
\DeclareMathOperator{\P}{\mathbb{P}}

\usepackage{multicol}

\usepackage[bibencoding = auto,
backend = biber,
sorting = none,
style=alphabetic]{biblatex}

\addbibresource{forecast_everything.bib}



% делаем короче интервал в списках
\setlength{\itemsep}{0pt}
\setlength{\parskip}{0pt}
\setlength{\parsep}{0pt}




\Newassociation{sol}{solution}{solution_file}
% sol --- имя окружения внутри задач
% solution --- имя окружения внутри solution_file
% solution_file --- имя файла в который будет идти запись решений
% можно изменить далее по ходу
\Opensolutionfile{solution_file}[all_solutions]
% в квадратных скобках фактическое имя файла

% магия для автоматических гиперссылок задача-решение
\newlist{myenum}{enumerate}{3}
% \newcounter{problem}[chapter] % нумерация задач внутри глав
\newcounter{problem}[section]

\newenvironment{problem}%
{%
\refstepcounter{problem}%
%  hyperlink to solution
     \hypertarget{problem:{\thesection.\theproblem}}{} % нумерация внутри глав
     % \hypertarget{problem:{\theproblem}}{}
     \Writetofile{solution_file}{\protect\hypertarget{soln:\thesection.\theproblem}{}}
     %\Writetofile{solution_file}{\protect\hypertarget{soln:\theproblem}{}}
     \begin{myenum}[label=\bfseries\protect\hyperlink{soln:\thesection.\theproblem}{\thesection.\theproblem},ref=\thesection.\theproblem]
     % \begin{myenum}[label=\bfseries\protect\hyperlink{soln:\theproblem}{\theproblem},ref=\theproblem]
     \item%
    }%
    {%
    \end{myenum}}
% для гиперссылок обратно надо переопределять окружение
% это происходит непосредственно перед подключением файла с решениями



\theoremstyle{definition}
\newtheorem{definition}{Определение}



\begin{document}
\begin{enumerate}
    \item Города A и B соединены железной дорогой длиной 60 км. 
    Поезда в обе стороны отправляются в каждый час одновременно, время в пути 1 час. 
    Стрелочник, живущий около железной дороги, любит подойти к окну, 
    дождаться первого проходящего мимо поезда и записать его направление. 
    Поезда обоих направлений в его записях встречаются одинаково часто. 

    \begin{enumerate}
        \item На каком расстоянии от ближайшего города он живёт?
        \item Сколько в среднем он ждёт поезда?
    \end{enumerate}
    
    \item Аня хватается за верёвку в форме окружности в произвольной точке.
    Боря берёт мачете и с завязанными глазами разрубает
    верёвку в двух случайных независимых местах. Аня забирает себе тот кусок,
    за который держится. Боря забирает оставшийся кусок. Вся верёвка имеет единичную длину.
    \begin{enumerate}
    \item Найдите вероятность того, что у Ани верёвка длиннее.
    \item Чему равен ожидаемый длина куска верёвки, доставшегося Ане?
    \end{enumerate}

    \item Неуловимый Джо подбрасывает правильную монетку бесконечное количество раз. 
    \begin{enumerate}
        \item Какая комбинация скорее всего выпадет раньше $HH$ или $TTT$? $HTT$ или $TTH$?
        \item Сколько бросков в среднем надо сделать до выпадения $HH$?
    \end{enumerate}

    \item В ювелирном магазине широко открыта дверь, а на открытой витрине лежат 150 колец и 420 кулончиков. 
    Пока продавщица болтает по телефону проказница-сорока залетает в магазин хватает по одному украшению наугад и относит к себе в гнездо
     до полного опустошения магазина.
    
    Какова вероятность того, что в какой-то момент времени кроме изначального в гнезде сороки колец будет столько же, сколько кулончиков?

    \item Андрей Абрикосов, Борис Бананов и Вова Виноградов играют одной командой в игру.
    В комнате три закрытых внешне неотличимых коробки: с абрикосами, бананами и виноградом.
    Общаться после начала игры они не могут, но могут заранее договориться о стратегии.
    Они заходят в комнату по очереди.
    Каждый из них может открыть две коробки по своему выбору.
    Перед следующим игроком коробки закрываются. Если Андрей откроет коробку абрикосами,
    Борис — с бананами, а Вова — с виноградом, то они выигрывают.
    Если хотя бы один из игроков не найдёт свой фрукт, то их команда проигрывает.
    
  Какова их оптимальная стратегия? А какова их наихудшая стратегия?

  \item Злобный Дракон поймал принцесс Настю и Сашу и посадил в разные башни.
  Перед каждой из принцесс Злобный Дракон подбрасывает один раз правильную монетку.
  А дальше даёт каждой из них шанс угадать, как выпала монетка у её подруги.
  Если хотя бы одна из принцесс угадает, то Злобный Дракон отпустит принцесс на волю.
  Если обе принцессы ошибутся, то они навсегда останутся у него в заточении.
  
  Подобная практика у Злобного Дракона исследователями была отмечена уже давно,
  поэтому принцессы имели достаточно времени договориться на случай вероятного похищения.
  
  Как следует поступать принцессам при подобных похищениях?

  \item Маша и Саша играют в быстрые шахматы. У них одинаковый класс игры и
  оба предпочитают играть белыми потому, что выигрывают белыми с вероятностью $0.7$.
  В ничью они никогда не играют.

  Партии играются до 10 побед. Первую партию Маша играет белыми.
  
  Она считает, что в следующей партии белыми должен играть тот,
  кто выиграл предыдущую партию. Саша считает, что ходить белыми нужно по очереди.

  При каком варианте правил у Маши больше шансы выиграть?
    


\end{enumerate}


\end{document}
